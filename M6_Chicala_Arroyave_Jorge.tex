\documentclass[a4paper,11pt]{article}
\usepackage[spanish]{babel}
\usepackage[utf8]{inputenc}
\begin{document}
\title{ESTRATEGIA DE SEGURIDAD PARA INTERNET Y EXTRANE EN EMPRESAS}
\author{Jorge Chicala Arroyave}
\date{\today}
\maketitle
\begin{bf}
URL del Repositorio: 
\end{bf}
https://github.com/jchicala/proyecto_final \\
\begin{bf}
\begin{center}
RESUMEN \\
\end{center}
\end{bf}
Con el crecimiento de la tecnología de información y la ubicuidad de los medios
de difusión y comunicación empresariales (Intranet y Extranet), es cada vez más
necesaria adecuadas medidas de seguridad de la información y en especial la 
implementación de un Sistema de Gestión de la Seguridad de la Información (SGSI).
 Una metodología adecuada en la cual se identifiquen los riesgos, se generen el 
diseño de seguridades y se asegure la memoria técnica de los procesos y el 
accionar de seguridad en el día a día, se hace necesaria, como un mecanismo para asegurar el cumplimiento de la seguridad de información.\newline   
\begin{bf}
Palabras claves:
\end{bf}
Intranet, Extranet, Seguridad, Octave, Magerit. \\
\begin{minipage}[t]{0.5\textwidth}
\end{minipage}
\cite{Monte2015}
\cite{OccupyTheWeb2019}
\bibliographystyle{apacite}
\bibliography{M5_Chicala_Arroyave_Jorge}
\end{document}
