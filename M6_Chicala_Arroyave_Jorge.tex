\documentclass[a4paper,11pt]{article}
\usepackage[spanish]{babel}
\usepackage[utf8]{inputenc}
\begin{document}
\title{ESTRATEGIA DE SEGURIDAD PARA INTERNET Y EXTRANE EN EMPRESAS}
\author{Jorge Chicala Arroyave}
\date{\today}
\maketitle
\begin{bf}
URL del Repositorio:
\end{bf}
https://github.com/jchicala/proyecto\_final
\begin{bf}
\begin{center}
RESUMEN \\
\end{center}
\end{bf}
Con el crecimiento de la tecnología de información y la ubicuidad de los medios
de difusión y comunicación empresariales (Intranet y Extranet), es cada vez más
necesaria adecuadas medidas de seguridad de la información y en especial la 
implementación de un Sistema de Gestión de la Seguridad de la Información (SGSI).
 Una metodología adecuada en la cual se identifiquen los riesgos, se generen el 
diseño de seguridades y se asegure la memoria técnica de los procesos y el 
accionar de seguridad en el día a día, se hace necesaria, como un mecanismo para asegurar el cumplimiento de la seguridad de información.\\   
\begin{bf}
Palabras claves:
\end{bf}
Intranet, Extranet, Seguridad, Octave, Magerit. \\
\begin{minipage}[t]{0.5\textwidth}
\begin{bf}
INTRODUCCION\\
\end{bf}
Nos estamos ahogando en información, pero hambrientos de conocimientos. 
John Naisbitt, Megatrends.\\
Las organizaciones y los individuos, actualmente, enfrentan una avalancha de 
datos e información, las cuales, muchas veces pueden resultar críticas en su 
naturaleza de agente diferenciador de competitividad, pero a la vez, es difícil 
de administrar y asegurar apropiadamente.\\
Actualmente las empresas están enfocando sus esfuerzos en administrar, de la 
mejor manera posible, sus activos más valiosos, los primarios que abarca la 
información y los secundarios que son aquellas herramientas de soporte para la 
información, entre los que se incluyen el software, hardware, comunicación, 
usuarios y la estructura que los soporta.\\
Cualquier sistema informático que maneje información y que permita la comunicación 
para compartir esta información, es vulnerable a ataques informáticos, es por 
este motivo que, el análisis de riesgo, basado en una metodología que permita la identificación y valoración del riesgo se convierte en una herramienta 
de ayuda para la definición e implementación de políticas y procedimientos de 
seguridad.\\   

\end{minipage}
\cite{Monte2015}
\cite{OccupyTheWeb2019}
\bibliographystyle{apacite}
\bibliography{M5_Chicala_Arroyave_Jorge}
\end{document}
